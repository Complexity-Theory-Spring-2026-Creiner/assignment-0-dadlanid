\documentclass[12pt]{article}
\usepackage[margin=1in]{geometry}
\usepackage{amsmath, amssymb} % math symbols and environments

\title{Assignment 0\\Darshan Dadlani} % <-- replace with your name

\begin{document}
\maketitle

\begin{enumerate}
\item By the chain rule from calculus, we have:
\begin{align}
\frac{d}{dx}\sin(x^2 + 6x) &= \cos(x^2 + 6x)\,\frac{d}{dx}(x^2 + 6x)\\
&= \cos(x^2 + 6x)\,(2x + 6)
\end{align}

\item DeMorgan's law says
\begin{equation}
\neg(A \cap B) \equiv \neg A \cup \neg B
\end{equation}

\item
\noindent\makebox[\linewidth]{%
\begin{tabular}{|c|c|c|}
\hline
A & B & $A \land B$\\
\hline
T & T & T\\
\hline
T & F & F\\
\hline
F & T & F\\
\hline
F & F & F\\
\hline
\end{tabular}
}

\item $\mathbb{R}$ denotes the real numbers. $\mathbb{N}$ denotes the natural numbers.
It is of course the case that $\mathbb{N} \subseteq \mathbb{R}$.
\end{enumerate}

\end{document}
